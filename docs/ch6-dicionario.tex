\chapter{Dicionário de Projeto}
\label{sec-dicionario}
\vspace{-1cm}

Esta seção apresenta as definições detalhadas das classes, descrevendo seus atributos e associações e servindo como um glossário do projeto. As definições são organizadas por subsistema, cada classe sendo apresentada em uma tabela separada. A coluna "Obr.?" indica com um "x" se o atributo é obrigatório (deve possuir um valor para se criar um objeto da classe).

\section{Gestão de Usuários}
\label{sec-dicionario-usuarios}

\begin{longtable}{|p{3.5cm}|c|c|p{8cm}|}
    \caption{Detalhamento da classe \emph{Usuario}.}
    \label{tbl-dicionario-usuario} \\\hline 
    
    \rowcolor{lightgray}
    \textbf{Propriedade} & \textbf{Tipo} & \textbf{Obr.?} & \textbf{Descrição} \\\hline
    \endfirsthead
    \hline
    \rowcolor{lightgray}
    \textbf{Propriedade} & \textbf{Tipo} & \textbf{Obr.?} & \textbf{Descrição} \\\hline
    \endhead
    
    id & UUID & x & Identificador único do usuário \\\hline
    nome & Texto & x & Nome completo do usuário \\\hline
    email & Email & x & Endereço de email do usuário \\\hline
    senha & Texto & x & Senha criptografada do usuário \\\hline
    dataCadastro & Data & x & Data de criação da conta \\\hline
    avatar & Imagem & & Foto de perfil do usuário \\\hline
    roteiros & Roteiro & & Lista de roteiros criados pelo usuário \\\hline
    avaliacoes & Avaliacao & & Lista de avaliações feitas pelo usuário \\\hline
	denuncias & Lista<Denuncia> & & Lista de denúncias feitas pelo Usuários \\\hline
    atracoesFavoritas & Atracao & & Lista de atrações marcadas como favoritas \\\hline
\end{longtable}

\section{Gestão de Atrações}
\label{sec-dicionario-atracoes}

\begin{longtable}{|p{3.5cm}|c|c|p{8cm}|}
    \caption{Detalhamento da classe \emph{Atracao}.}
    \label{tbl-dicionario-atracao} \\\hline 
    
    \rowcolor{lightgray}
    \textbf{Propriedade} & \textbf{Tipo} & \textbf{Obr.?} & \textbf{Descrição} \\\hline
    \endfirsthead
    \hline
    \rowcolor{lightgray}
    \textbf{Propriedade} & \textbf{Tipo} & \textbf{Obr.?} & \textbf{Descrição} \\\hline
    \endhead
    
    id & UUID & x & Identificador único da atração \\\hline
    nome & Texto & x & Nome da atração turística \\\hline
    descricao & Texto & x & Descrição detalhada da atração \\\hline
    endereco & Endereco & x & Localização da atração \\\hline
    categoria & Categoria & x & Categoria da atração (ex: Museu, Parque) \\\hline
    imagens & Lista<Imagem> & & Galeria de imagens da atração \\\hline
    horarioFuncionamento & Texto & x & Horários de funcionamento \\\hline
    precoMedio & Decimal & & Preço médio do ingresso \\\hline
    avaliacoes & Avaliacao & & Lista de avaliações da atração \\\hline
	denuncias & Lista<Denuncia> & & Lista de denúncias sobre a atração \\\hline
    ofertas & Oferta & & Lista de ofertas relacionadas \\\hline
\end{longtable}

\section{Gestão de Roteiros}
\label{sec-dicionario-roteiros}

\begin{longtable}{|p{3.5cm}|c|c|p{8cm}|}
    \caption{Detalhamento da classe \emph{Roteiro}.}
    \label{tbl-dicionario-roteiro} \\\hline 
    
    \rowcolor{lightgray}
    \textbf{Propriedade} & \textbf{Tipo} & \textbf{Obr.?} & \textbf{Descrição} \\\hline
    \endfirsthead
    \hline
    \rowcolor{lightgray}
    \textbf{Propriedade} & \textbf{Tipo} & \textbf{Obr.?} & \textbf{Descrição} \\\hline
    \endhead
    
    id & UUID & x & Identificador único do roteiro \\\hline
    titulo & Texto & x & Título do roteiro \\\hline
    descricao & Texto & x & Descrição detalhada do roteiro \\\hline
    criador & Usuario & x & Usuário que criou o roteiro \\\hline
    dataCriacao & Data & x & Data de criação do roteiro \\\hline
    duracaoEstimada & Inteiro & x & Duração estimada em dias \\\hline
    atracoes & Lista<Atracao> & x & Lista de atrações incluídas no roteiro \\\hline
    avaliacoes & Avaliacao & & Lista de avaliações do roteiro \\\hline
    categoria & Categoria & x & Categoria do roteiro (ex: Família, Aventura) \\\hline
	denuncias & Lista<Denuncia> & & Lista de denúncias sobre o roteiro \\\hline
    status & Status & x & Status do roteiro (Público/Privado) \\\hline
\end{longtable}

\section{Gestão de Avaliações}
\label{sec-dicionario-avaliacoes}

\begin{longtable}{|p{3.5cm}|c|c|p{8cm}|}
    \caption{Detalhamento da classe \emph{Avaliacao}.}
    \label{tbl-dicionario-avaliacao} \\\hline 
    
    \rowcolor{lightgray}
    \textbf{Propriedade} & \textbf{Tipo} & \textbf{Obr.?} & \textbf{Descrição} \\\hline
    \endfirsthead
    \hline
    \rowcolor{lightgray}
    \textbf{Propriedade} & \textbf{Tipo} & \textbf{Obr.?} & \textbf{Descrição} \\\hline
    \endhead
    
    id & UUID & x & Identificador único da avaliação \\\hline
    autor & Usuario & x & Usuário que fez a avaliação \\\hline
    nota & Inteiro & x & Nota de 1 a 5 \\\hline
    comentario & Texto & x & Texto da avaliação \\\hline
    dataAvaliacao & Data & x & Data em que a avaliação foi feita \\\hline
    tipoAvaliacao & Tipo & x & Tipo (Atração/Roteiro) \\\hline
    itemAvaliado & UUID & x & ID do item avaliado (Atração ou Roteiro) \\\hline
    denuncias & Lista<Denuncia> & & Lista de denúncias sobre a avaliação \\\hline
    fotos & Lista<Imagem> & & Fotos anexadas à avaliação \\\hline
\end{longtable}

\section{Gestão de Ofertas}
\label{sec-dicionario-ofertas}

\begin{longtable}{|p{3.5cm}|c|c|p{8cm}|}
    \caption{Detalhamento da classe \emph{Oferta}.}
    \label{tbl-dicionario-oferta} \\\hline 
    
    \rowcolor{lightgray}
    \textbf{Propriedade} & \textbf{Tipo} & \textbf{Obr.?} & \textbf{Descrição} \\\hline
    \endfirsthead
    \hline
    \rowcolor{lightgray}
    \textbf{Propriedade} & \textbf{Tipo} & \textbf{Obr.?} & \textbf{Descrição} \\\hline
    \endhead
    
    id & UUID & x & Identificador único da oferta \\\hline
    titulo & Texto & x & Título da oferta \\\hline
    descricao & Texto & x & Descrição detalhada da oferta \\\hline
    atracao & Atracao & x & Atração relacionada à oferta \\\hline
    preco & Decimal & x & Preço promocional \\\hline
    dataInicio & Data & x & Data de início da oferta \\\hline
    dataFim & Data & x & Data de término da oferta \\\hline
    status & Status & x & Status da oferta (Ativa/Inativa) \\\hline
    denuncias & Lista<Denuncia> & & Lista de denúncias sobre a oferta \\\hline
\end{longtable}

\section{Classes Auxiliares}
\label{sec-dicionario-auxiliares}

\begin{longtable}{|p{3.5cm}|c|c|p{8cm}|}
    \caption{Detalhamento da classe \emph{Endereco}.}
    \label{tbl-dicionario-endereco} \\\hline 
    
    \rowcolor{lightgray}
    \textbf{Propriedade} & \textbf{Tipo} & \textbf{Obr.?} & \textbf{Descrição} \\\hline
    \endfirsthead
    \hline
    \rowcolor{lightgray}
    \textbf{Propriedade} & \textbf{Tipo} & \textbf{Obr.?} & \textbf{Descrição} \\\hline
    \endhead
    
    logradouro & Texto & x & Nome da rua \\\hline
    numero & Texto & x & Número do local \\\hline
    complemento & Texto & & Informações adicionais \\\hline
    bairro & Texto & x & Nome do bairro \\\hline
    cidade & Texto & x & Nome da cidade \\\hline
    estado & Texto & x & Nome do estado \\\hline
    cep & CEP & x & Código postal \\\hline
    coordenadas & Coordenadas & x & Latitude e longitude \\\hline
\end{longtable}

\begin{longtable}{|p{3.5cm}|c|c|p{8cm}|}
    \caption{Detalhamento da classe \emph{Denuncia}.}
    \label{tbl-dicionario-denuncia} \\\hline 
    
    \rowcolor{lightgray}
    \textbf{Propriedade} & \textbf{Tipo} & \textbf{Obr.?} & \textbf{Descrição} \\\hline
    \endfirsthead
    \hline
    \rowcolor{lightgray}
    \textbf{Propriedade} & \textbf{Tipo} & \textbf{Obr.?} & \textbf{Descrição} \\\hline
    \endhead
    
    id & UUID & x & Identificador único da denúncia \\\hline
    denunciante & Usuario & x & Usuário que realizou a denúncia \\\hline
    tipoDenuncia & Enum & x & Categoria da denúncia (Conteúdo Inadequado, Informação Incorreta, Spam, Ofensivo, Outros) \\\hline
    descricao & Texto & x & Descrição detalhada do motivo da denúncia \\\hline
    dataDenuncia & Data & x & Data e hora em que a denúncia foi realizada \\\hline
    status & Enum & x & Status da denúncia (Pendente, Em Análise, Resolvida, Rejeitada) \\\hline
    tipoItem & Enum & x & Tipo do item denunciado (Avaliação, Oferta) \\\hline
    itemDenunciado & UUID & x & Identificador do item que está sendo denunciado \\\hline
    dataConclusao & Data & & Data em que a denúncia foi analisada e concluída \\\hline
    parecerModerador & Texto & & Parecer do moderador sobre a denúncia \\\hline
    evidencias & Lista<Arquivo> & & Arquivos ou prints que comprovem o motivo da denúncia \\\hline
\end{longtable}
